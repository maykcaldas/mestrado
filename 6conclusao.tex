\chapter{Conclusão}

Simulações de dinâmica molecular foram efetuadas para os dendrímeros PAMAM e PPI de geração $1$ à $5$ e $1$ à $6$, respectivamente, utilizando o campo de força \textit{GROMOS-compatible} 2016H66.
As topologias e configurações iniciais das moléculas dendríticas foram criadas com o PyPolyBuilder de maneira automatizada, sendo necessário o mínimo de esforço.
O pacote de simulação Gromacs-5.1.4 foi utilizado seguindo as sugestões de \textit{setup} de Gonçalvez \textit{et al}, mas alterações do método de avaliação das interações de Lennard-Jones não se mostraram de grande impacto no sistema.
Para o PPI, diferentes taxas de protonação foram consideradas para o pH neutro e foi observado que os resultados dessa mudança são bem sutís.
Mas o modelo considerado mais correto pela literatura foi o utilizado nas argumentações da presente dissertação.
As propriedades raio de giro $R_g$, razões de aspecto, asphericidade $\delta$ e funções de distribuição radial foram calculadas e exaustivamente comparadas a dados experimentais, teóricos e computacionais relatados na literatura.
Foi visto que os dados quantitativos de raio de giro $R_g$ calculados pelo 2016H66 se mostraram muitas vezes mais adequados aos resultados experimentais.
Também foi mostrado que os valores relatados de SANS na literatura têm resultados questionáveis e que novos estudos de SAXS e SANS levando em conta o efeito do pH devem ser efetuados para sanar a dúvida em relação ao inchaço dos dendrímeros com alterações no pH ser um artefato das simulações ou modelos ajustados equivocadamente aos dados experimentais.
Por ter melhor se adequado quantitativamente aos experimentos e qualitativamente às predições teóricas, acreditamos que o 2016H66 é um bom campo de força e, ao menos para o cálculo de propriedades estruturais, está devidamente validado para a simulação de dendrímeros e deve ser utilizado em trabalhos futuros nessa linha de pesquisa.

\chapter{Perspectivas futuras}

Tendo sido o campo de forças GROMOS-compatible 2016H66\cite{Horta2016} validado para propriedades estruturais, estudos específicos podem ser iniciados.
Primeiramente a complexação de dendrímeros com fármacos de interesse deve ser estudado para que a validação completa, ou seja, também em propriedades energéticas possa ser conferida.
Após isso, como sugerido anteriormente, os dendrímeros são candidatos promissores como carreadores de fármacos.
Nessa linha de pensamento, novas simulações devem ser efetuadas para o estudo e entendimento desses sistemas.
Futuramente, esse projeto pode seguir as seguintes idéias:
$(i)$   Efetuar simulações de dinâmica molecular dos dendrímeros complexado com moléculas orgânicas de diferentes grupos funcionais e tamanhos de cadeia e avaliar a energia livre de complexação e carregamento máximo. Esse estudo visa obter o entendimento sistemático sobre quais grupos funcionais interagem melhor com o dendrímero e qual o efeito do tamanho de cadeia e do pH do meio nessa complexação;
$(ii)$  Devido à alta toxicidade, provavelmente o dendrímero não poderá ser utilizado $in \; vivo$ na sua forma não-funcionalizada. Por isso, também é interessante o estudo da complexação de fármacos com dendrímeros PEGylados. Espera-se, nesse estudo, entender os efeitos da PEGylação na dinâmica de liberação de fármacos;
$(iii)$ Também, como reportado, o campo de força GROMOS foi o único capaz de reproduzir o resultado experimental em que o dendrímero têm uma adsorção espontânea em membranas lipídicas. Dessa forma, é interessante fazer uso desse campo de força para avaliar a adsorção na membrana e a energia livre associada à permeação do dendrímero pela mesma.

\pagebreak
